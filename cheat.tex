\documentclass[11pt,a4paper]{article} 

\usepackage[utf8]{inputenc}
\usepackage[T1]{fontenc}
\usepackage[french]{babel}
\usepackage{graphicx} % images3
\usepackage{lmodern}
\usepackage{lscape}
\usepackage{textcomp}
\usepackage{threeparttable}
\usepackage{xcolor}
\usepackage{amssymb}% http://ctan.org/pkg/amssymb
\usepackage{pifont}% http://ctan.org/pkg/pifont
\newcommand{\cmark}{\color{olive}\ding{51}}%
\newcommand{\xmark}{\color{gray}\ding{55}}%
\usepackage[margin=0.75in]{geometry} % Adjusts the margins
\usepackage{amsthm}
\theoremstyle{plain}
\newtheorem*{theorem}{Définition}

\begin{document}
\begin{landscape}
	
\section{Groupes}
\begin{theorem}
	Un \textbf{magma} est un ensemble muni d'une loi de composition interne.
\end{theorem}

\begin{table}[h]
\centering
\begin{threeparttable}[h]
\begin{tabular}{l|cccc}
	\hline
	& Associatif & Unifère\tnote{1}  & Symétrique\tnote{2} & Commutatif
	\\\hline
	Magma & \xmark & \xmark & \xmark & \xmark
	\\
	Demi-groupe & \cmark & \xmark & \xmark & \xmark
	\\
	Monoïde & \cmark & \cmark & \xmark & \xmark
	\\
	Groupe & \cmark & \cmark & \cmark & \xmark
	\\
	Groupe abélien& \cmark & \cmark & \cmark & \cmark
	\\\hline
\end{tabular}
\begin{tablenotes}\footnotesize
	\item[1] Existence d'un neutre.
	\item[2] Tout élément a un inverse.
\end{tablenotes}
\end{threeparttable}
\caption{Groupes}
\end{table}

\section{Anneaux}
Soit la structure algébrique $(E,+,\times ,0,1)$.
%L'opérateur $+$ est \textbf{commutatif}.

\begin{table}[h]
\centering
\begin{threeparttable}[h]
\begin{tabular}{l|cc|c|cccccc}
	\hline
	& $(E,\times,1)$ & $(E,+,0)$   & Distributif & Unitaire\tnote{1} & \multicolumn{2}{c}{Symétrique} & \multicolumn{2}{c}{Commutatif} & Sans div. de 0\\
	&&&\& absorbant\tnote{3} & & ~$\times$\tnote{2}~ & ~+~ &  ~~$\times$~~ & ~~+~~ &
	\\\hline
	Pseudo-anneau & Demi-groupe & Groupe commutatif &  \cmark &  \xmark & \xmark &\cmark & \xmark & \cmark & \xmark
	\\
	Demi-anneau & Monoïde & Monoïde commutatif\tnote{3} &  \cmark &  \cmark & \xmark &\xmark & \xmark & \cmark & \xmark
	\\
	Anneau (unitaire) & Monoïde & Groupe commutatif &  \cmark &  \cmark & \xmark &\cmark & \xmark & \cmark & \xmark
	\\
	Anneau commutatif & Monoïde commutatif & Groupe commutatif & \cmark &  \cmark & \xmark &\cmark & \cmark & \cmark & \xmark
	\\
	Anneau intègre\tnote{4} & \begin{tabular}{c}Monoïde commutatif\\sans diviseur de zéro\end{tabular} & Groupe commutatif
	&  \cmark & \cmark & \xmark &\cmark & \cmark & \cmark & \cmark
	\\
	Corps gauche & Groupe\tnote{2} & Groupe commutatif
		&  \cmark & \cmark & \cmark &\cmark & \xmark & \cmark & \cmark
	\\
	Corps (commutatif) & Groupe commutatif\tnote{2} & Groupe commutatif
	&  \cmark & \cmark & \cmark &\cmark & \cmark & \cmark & \cmark
	\\\hline
\end{tabular}
\begin{tablenotes}\footnotesize
	\item[1] Existence d'un neutre au $\times$. Aussi dit unifère.
% 	\item[2] Tout élément a un inverse par $+$ et $\times$.
	\item[2] Sur $E \setminus \{0\}$.
	\item[3] Il faut prouver séparément que le 0 est absorbant pour les demi-anneaux.
	\item[4] Ou anneau nul (l'anneau intègre ne peut être nul).
\end{tablenotes}
\end{threeparttable}
\caption{Anneaux}
\end{table}
	
\end{landscape}	
\end{document}
